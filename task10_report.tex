\documentclass[a4paper,dvipdfmx]{jsarticle}
\usepackage{graphicx}
\usepackage{fancybox}
\usepackage{ascmac}
\usepackage{amsmath}
\usepackage{amsfonts}
\usepackage{comment}
\usepackage{here}
\usepackage{listings,jlisting}
\usepackage[a4paper,margin=2cm,truedimen]{geometry}
\usepackage{amssymb}
\usepackage[svgnames]{xcolor}

  \definecolor{diffstart}{named}{Grey}
  \definecolor{diffincl}{named}{Green}
  \definecolor{diffrem}{named}{OrangeRed}

  \lstdefinelanguage{diff}{
    basicstyle=\ttfamily\small,
    morecomment=[f][\color{diffstart}]{@@},
    morecomment=[f][\color{diffincl}]{+\ },
    morecomment=[f][\color{diffrem}]{-\ },
  }

\newcommand{\image}[2]{
    \begin{figure}[H]
        \begin{center}
        \includegraphics[width=14cm]{#2}
        \end{center}
        \caption{#1}
        \label{#1}
    \end{figure}
}

\newcommand{\code}[1]{\texttt{#1}}

\lstset{
	language=[Objective]Caml,
    basicstyle=\ttfamily,
    keywordstyle=\color[RGB]{33,74,135}\bfseries,
    stringstyle=\color[RGB]{79,153,5},
    commentstyle=\color[RGB]{143,89,2}\itshape,
    numberstyle=\footnotesize,
    numbers=left,
    stepnumber=1,
    numbersep=15pt,
    backgroundcolor=\color[RGB]{251,251,251},
    frame=single,
    frameround=ffff,
    framesep=5pt,
    rulecolor=\color[RGB]{148,150,152}, 
    breaklines=true,
    breakautoindent=true,
    breakatwhitespace=true,
    breakindent=25pt,
    showspaces=false,
    showstringspaces=false,
    showtabs=false,
    tabsize=2,
    captionpos=b,
    linewidth=\textwidth,
}

\begin{document}

\title{計算機科学実験レポート3 タスク10}
\author{1029-28-9483 勝田 峻太朗}
\date{\today}

\maketitle

\section*{課題10 (任意: バックエンドの移植)}

\begin{quotation}
ARMアセンブリコードではなくC言語コードへ変換するコード生成器を作成し,性能比較を行いなさい.
\end{quotation}


\subsection*{C言語への変換の方針}

\subsubsection*{代入文}

\begin{lstlisting}
<type_of_src> dst = src
\end{lstlisting}

ただし,\lstinline{<type_of_src>} には,\lstinline{"closure"(クロージャ), "int"(数), "int*"(配列), ""(すでに定義されている)}
のいずれかが入る.

例として, ソースコード\ref{ml-assignment}の変換結果は ソースコード\ref{c-assignment} のようななる.

\begin{lstlisting}[label=ml-assignment, caption=MLコードの例]
let a = 3 in let b = a in 0;;
\end{lstlisting}

\begin{lstlisting}[language=C, label=c-assignment, caption=変換後のC言語]
int main(){

int var00 = 3;
int var10 = var00;
int var30 = 0;
printf("%d\n", var30);
return 0;
}
\end{lstlisting}


\subsubsection*{labelとgoto文}

\begin{lstlisting}
<labelname>:;
<some operation>
<some operation>
goto <labelname>;
\end{lstlisting}

のような形で,C言語のgoto文を用いて実装する.

\subsubsection*{クロージャと関数呼び出し}

クロージャは, 以下のような struct を用いて実装した.

\begin{lstlisting}[language=C, caption=C言語でのクロージャの実装]
typedef struct{
    int (*f)(const int*, const int);
    int* vars;
    int length;

} closure;
\end{lstlisting}

$f$は,第1要素目の関数ポインタ(クロージャの第0要素)を表し,$vars$は,スコープ外変数の列(クロージャの第1要素以降)を表す.
$length$は,$vars$の長さを保持する.(ただし実装では用いられていない.)

また, 混乱を避けるため,クロージャの第1要素は$vars$の第0要素ではなく,第1要素に格納される.

例えば,ソースコード\ref{ml-closure-ex}は,ソースコード\ref{c-closure-ex}のように変換される.

\begin{lstlisting}[caption=クロージャを含むMLコード, label=ml-closure-ex]
let a = 1 in let b = 2 in let rec f x = x + a + b in f 5;;
\end{lstlisting}

\begin{lstlisting}[caption=クロージャの変換例, language=C, label=c-closure-ex]
#include <stdio.h>
#include <stdlib.h>

typedef struct{
    int (*f)(const int*, const int);
    int* vars;
    int length;

} closure;

int _b__recf10(const int *param_0, const int param_1){

int var02 = param_0[2];
int var12 = param_1;
int var21 = param_0[1];
int var32 = var12 + var02;
int var51 = var32 + var21;
return var51;
}

int main(){

int var03 = 1;
int var13 = 2;
closure var22;
var22.f = _b__recf10;
var22.length = 3;
int params[3];
params[1] = var13;
params[2] = var03;
var22.vars = params;
closure var33 = var22;
int var40 = 5;
int (*var52)(const int*, const int);
var52 = var33.f;
int var70 = var52(var33.vars, var40);
printf("%d\n", var70);
return 0;
}
\end{lstlisting}


\subsection*{C言語の構造の定義}

C言語上で使用する命令を, $c\_spec.ml$において,定義した.

\begin{lstlisting}[caption=c\_spec.mlにおけるC言語の構造定義]
(* variable names *)
type id = string
type label = string
type imm = int
type binop = Syntax.binOp

type ty = 
  | Int
  | Closure 
  | Tuple 
  | Defined

type op =
  | Var of id
  | Imm of imm

type exp = 
  | Decl of ty * id * op (* int id = op *)
  | Exp of op (* operand *)
  | Bin of id * binop * exp * exp 
  | If of op * exp 
  | Write of id * int * op (* id[i] = op *)
  | Return of op (* return op *)
  | Print of op
  | Read of id * op * int (* id = op[int] *)
  | Label of label
  | Goto of label
  | Call of id * id * id * op (* id = id(id.vars, x) *)
  | DeclareTuple of id * int
  | SetTupleValue of id * int * op
  | DeclareClosure of id (* closure aru_closure; *)
  | SetClosurePointer of id * label (* aru_closure.f = b__recf00 *)
  | SetClosureLength of id * int (* aru_closure.length = 2 *)
  | DeclareClosureParams of int (* int params[i]; *)
  | StoreClosureParams of int * op (* params[i] = op *)
  | SetClosureParams of id (* aru_closure.vars = params *)
  | DeclarePointer of id
  | AssignPointer of id * id (* id = id.f *)
  | Exit

type funct = Funct of id * (id list) * (exp list)
\end{lstlisting}

\subsection*{VMコードからC言語への変換}

VMコードから\lstinline{c_spec.ml}で示したC言語の構造への変換は\lstinline{c_of_decl}関数を用いて行った.

ただし, \lstinline{ref}型の値は以下の内容を持つ.

\begin{description}

\item[var\_assoc] \hfill

$Vm.id$か$id$への変換を保持する.

\item[closure\_var] \hfill

中身がクロージャである$id$集合を保持する.

\item[tuple\_var]  \hfill

中身が(クロージャではなく)tupleである$id$集合を保持する.

\item[defined\_var] \hfill

すでに定義された変数集合を保持する.

\end{description}

\begin{lstlisting}[caption=backend.ml - C言語への変換]
let c_of_decl (Vm.ProcDecl(lbl, local_var, instr_list)): funct =
  (* helper definitions and functions *)
  let var_assoc = ref (MyMap.empty: (V.id, id) MyMap.t) in
  let closure_var = ref (MySet.empty: id MySet.t) in
  let tuple_var = ref (MySet.empty: id MySet.t) in
  let defined_var = ref (MySet.empty: id MySet.t) in
  let id_is_closure (id: id) = MySet.member id !closure_var in
  let id_is_tuple (id: id) = MySet.member id !tuple_var in
  let id_is_defined (id:  id) = MySet.member id !defined_var in
  let op_is_closure = function
    | Var id -> id_is_closure id
    | Imm _ -> false in
  let op_is_tuple = function
    | Var id -> id_is_tuple id
    | Imm _ -> false in
  let append_closure id = closure_var := MySet.insert id !closure_var in
  let append_tuple id = tuple_var := MySet.insert id !tuple_var in
  let append_defined id = defined_var := MySet.insert id !defined_var in
  let convert_id id = 
    match MyMap.search id !var_assoc with
    | Some id' -> id'
    | None -> let id' = fresh_var (string_of_int id) in
      var_assoc := MyMap.append id id' !var_assoc;
      id' in
  let id_of_op op = 
    match op with 
    | V.Local id -> convert_id id
    | _ -> err "id_of_exp: unexpected input" in
  let convert_op op = 
    match op with
    | V.Param i -> if i = 0 then Var param0_name
      else Var param1_name
    | V.Local id -> Var (convert_id id)
    | V.IntV i -> Imm i
    | V.Proc l -> Var ("unexpected_" ^ l) in
  (* end helper definition and functions *)
  let rec c_of_instr instr = 
    match instr with 
    | V.Move(id, op) -> 
      let id' = convert_id id in
      let op' = convert_op op in
      let id_was_defined = id_is_defined id' in
      let is_closure = op_is_closure op' in
      let is_tuple = op_is_tuple op' in
      if is_closure then append_closure id';
      if is_tuple then append_tuple id';
      append_defined id';
      let ty = if id_was_defined then Defined 
        else (if is_closure then Closure else if is_tuple then Tuple else Int) in
      [Decl(ty, convert_id id, convert_op op)]
    | V.BinOp(id, binop, op1, op2) -> [Bin(convert_id id, binop, Exp(convert_op op1), Exp(convert_op op2))]
    | V.Label l -> [C_spec.Label l]
    | V.BranchIf(op, l) -> [If(convert_op op, Goto(l))]
    | V.Goto l -> [Goto l]
    | V.Call(dest, op, opl) -> 
      (match opl with
       | closure:: x:: [] -> [Call(convert_id dest, id_of_op op, id_of_op closure, convert_op x)]
       | _ -> err "unexpected function call")
    | V.Return op -> if lbl = "_toplevel" 
      then [Print(convert_op op); Exit]
      else [Return(convert_op op)]
    | V.Malloc(id, opl) ->
      (match opl with
       | pointer:: vars ->
         if is_vm_proc pointer then (* if tuple is closure *)
           let closure_name = convert_id id in
           append_closure closure_name;
           append_defined closure_name;
           let funct_name = label_of_proc pointer in
           let var_len = List.length vars in
           [DeclareClosure closure_name; SetClosurePointer(closure_name, funct_name);
            SetClosureLength (closure_name, var_len + 1); DeclareClosureParams(var_len + 1)] @
           (List.mapi (fun i op -> StoreClosureParams(i+1, convert_op op)) vars) @
           [SetClosureParams closure_name;]
         else (* if tuple is not closure (a normal tuple) *)
           let len = List.length opl in
           let id' = convert_id id in
           append_tuple id';
           [DeclareTuple(id', len)] @
           List.mapi (fun i op -> SetTupleValue(id', i, convert_op op)) opl
       | _ -> err "undefined")
    | V.Read(id, op, i) -> 
      if op_is_closure (convert_op op) then 
        (if i = 0 
         then [DeclarePointer(convert_id id); AssignPointer(convert_id id, id_of_op op)]
         else [Read(convert_id id, convert_op op, i)])
      else [Read(convert_id id, convert_op op, i)]
    | V.BEGIN _ -> err "error"
    | V.END _ -> err "error" in
  let instrs = List.concat (List.map c_of_instr instr_list) in
  if lbl = "_toplevel" 
  then Funct("main", [], instrs)
  else Funct(lbl, [param0_name; param1_name], instrs)


let convert_c = List.map c_of_decl 
\end{lstlisting}


\subsection*{コマンドラインオプション}

\begin{description}
\item[-C] \hfill
このオプションをつけると, 生成されたCプログラムが\lstinline{gcc}によってコンパイルされる. \lstinline{./a.out} を実行することにより,プログラムを実行できる.

\item[-o filename] \hfill
結果を\lstinline{<filename>}に書き込む.

\begin{lstlisting}[caption=実行例, language=bash]
$ ./minimlc -C -o sigma.c
# loop v = (1, 0) in
if v.1 < 101 then
  recur (v.1 + 1, v.1 + v.2)
else
  v.2;;
compile c program in sigma.c => success
# ^C
$./a.out
5050

\end{lstlisting}
\end{description}






\end{document}